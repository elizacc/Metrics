\documentclass[12pt]{article}
\usepackage[svgnames]{xcolor}
\usepackage{tikz} 
\usetikzlibrary{arrows} 
\usetikzlibrary{calc,positioning,fit,backgrounds} 
\usetikzlibrary{decorations.pathreplacing,calc}
\usepackage{pgfplots}
\usepackage[russian]{babel}
\parskip=6pt
\usepackage{multicol}
\usepackage{amssymb, amsmath}
\oddsidemargin=-0,5cm
\textwidth=18cm
\topmargin=-3cm
\pagestyle{plain}
\textheight=25.5cm

\newcommand{\e}{\mathbb{E}}
\newcommand{\p}{\mathbb{P}}
\newcommand{\n}{\mathbb{N}}
\newcommand{\id}{\mathbb{I}}
\newcommand{\re}{\mathbb{R}}
\DeclareMathOperator{\plim}{plim}
\DeclareMathOperator{\var}{Var}
\DeclareMathOperator{\svar}{sVar}
\renewcommand{\epsilon}{\varepsilon}
\newcommand{\msum}{\sum\limits_1^n}
\newcommand{\isum}{\sum\limits_i}
\newcommand{\jsum}{\sum\limits_j}

\begin{document}
	
\begin{center}
Конспект по метрике к КР-1
\end{center}

1) Парная регрессия: $\e (Y \, | \, X = x) = \beta_0 + \beta_1 x$

$Y_i = \beta_0 + \beta_1 X_i + \epsilon_i$ \qquad
\parbox{3cm}{$Y_i$ -- зависимая; $X_i$ -- регрессор; $\epsilon_i$ -- навязка}

$Y_i = \hat{\beta}_0 + \hat{\beta}_1 X_i + \underset{\text{остаток}}{e_i} \qquad \qquad RSS = \sum\limits_1^n e_i^2 = \sum\limits_1^n (Y_i - \hat{Y}_i)^2$

$\hat{Y}_i = \hat{\beta}_0 + \hat{\beta}_1 X_i$

2) МНК: $RSS \rightarrow min$

$FOC: \left\{
\begin{array}{rcl}
n \hat{\beta}_0 + \sum\limits_1^n X_i \hat{\beta}_1 & = &\sum\limits_1^n Y_i\\[6mm]
\sum\limits_1^n X_i \hat{\beta}_0 + \sum\limits_1^n X_i^2 \hat{\beta}_1 & = & \sum\limits_1^n X_i Y_i\\
\end{array}
\right.
\Rightarrow
\begin{array}{rcl}
\hat{\beta}_1 & = &  \frac{\sum\limits_1^n (X_i - \bar{X})(Y_i - \bar{Y})}{\sum\limits_1^n(X_i - \bar{X})^2} = \frac{sCov(X, Y)}{\svar(X)}\\[6mm]
\hat{\beta}_0 & = & \bar{Y} - \hat{\beta}_1 \bar{X}\\
\end{array}
$

3) Леммы:
\begin{enumerate}
	\item Линия регрессии проходит через точку $(\bar{X}, \bar{Y})$, т.е. $\bar{Y} = \hat{\beta}_0 + \hat{\beta}_1 \bar{X}$
	
	\textbf{Док-во}: $\hat{\beta}_0 = \bar{Y} - \hat{\beta}_1 \bar{X} \Rightarrow$ \fbox{$\bar{Y} = \hat{\beta}_0 + \hat{\beta}_1 \bar{X}$}
	
	\item $\sum\limits_1^n e_i = 0$
	
	\textbf{Док-во}:
	
	$e_i = Y_i - \hat{\beta}_0 - \hat{\beta}_1 X_i$\\[3mm]
	$\sum\limits_1^n e_i = \sum\limits_1^n Y_i - n \hat{\beta}_0 - \hat{\beta}_1 \msum X_i$
	
	$\frac{1}{n} \sum\limits_1^n e_i = \bar{Y} - \hat{\beta}_0 - \hat{\beta}_1 \bar{X} = 0 \Rightarrow$ \framebox{$\sum\limits_1^n e_i = 0$}
	
	\item $\msum Y_i = \msum \hat{Y}_i$
	
	\begin{minipage}{0.5\textwidth}
	\textbf{Док-во}:
	
	$\msum e_i = \msum Y_i - \msum \hat{Y}_i = 0 \Rightarrow$ \fbox{$\msum Y_i = \msum \hat{Y}_i$}
	\end{minipage}
	\begin{minipage}{0.4\textwidth}
	\textbf{Следствиe}:
	
	$\frac{1}{n} \msum e_i = \bar{Y} - \bar{\hat{Y}} = 0 \Rightarrow$ \fbox{$\bar{Y} = \bar{\hat{Y}}$}
	\end{minipage}
	\item $\msum X_i e_i = 0$
	
	\textbf{Док-во}:
	
	$\frac{\partial RSS}{\partial \hat{\beta}_1} = -2 \msum (Y_i - \hat{\beta}_0 - \hat{\beta}_1 X_i) X_i = 0 \Rightarrow$ \fbox{$\msum X_i e_i = 0$}
	\item $\msum \hat{Y}_i e_i = 0$
	
	\textbf{Док-во}:
	
	$\msum \hat{Y}_i e_i = \msum (\hat{\beta}_0 + \hat{\beta}_1 X_i) e_i = \hat{\beta} \msum e_i + \hat{\beta}_1 \msum X_i e_i = 0 \Rightarrow$ \fbox{$\msum \hat{Y}_i e_i = 0$}
\end{enumerate}

4) $Y_i = \hat{Y}_i + e_i; \; \bar{Y} = \bar{\hat{Y}} \Rightarrow Y_i - \bar{Y} = \hat{Y}_i - \bar{\hat{Y}} + e_i$

$\msum (Y_i - \bar{Y})^2 = \msum (\hat{Y}_i - \bar{\hat{Y}})^2 + 2 (\msum \hat{Y}_i e_i - \bar{\hat{Y}} \msum e_i) + \msum e_i^2$

$\msum (Y_i - \bar{Y})^2 = \msum (\hat{Y}_i - \bar{\hat{Y}})^2 + \msum e_i^2$

$TSS = ESS + RSS$

5) Коэффициент детерминации $R^2$

$R^2 = \frac{ESS}{TSS}; \; x_i = X_i - \bar{X}; \; y_i = Y_i - \bar{Y}; \; \bar{\hat{Y}} = \hat{\beta}_0 + \hat{\beta}_1 \bar{X}$

$R^2 = \frac{\msum (\hat{Y}_i - \bar{\hat{Y}})^2}{\msum y_i^2} = \frac{\msum (\hat{\beta}_0 + \hat{\beta}_1 X_i - \hat{\beta}_0 - \hat{\beta}_1 \bar{X})^2}{\msum y_i^2} = \frac{\msum (\hat{\beta}_1 x_i)^2}{\msum y_i^2} = \frac{\hat{\beta}_1^2 \msum x_i^2}{\msum y_i^2} = \frac{\msum (x_i y_i)^2 \cdot \msum x_i^2}{(\msum x_i^2)^2 \cdot \msum y_i^2} =$

$= \frac{\msum (x_i y_i)^2}{\msum x_i^2 \cdot \msum y_i^2} = \frac{sCov^2(X,Y)}{\svar(X) \cdot \svar(Y)} = sCorr^2(X,Y)$

6) Регрессия без свободного члена

\begin{minipage}{0.18\textwidth}
$Y_i = \beta_1 X_i + \epsilon_i$\\[2mm]	
$\hat{Y}_i = \hat{\beta}_1 X_i$\\[2mm]		
$e_i = Y_i - \hat{\beta}_1 X_i$
\end{minipage}
\begin{minipage}{0.35\textwidth}
$RSS \rightarrow \min: \hat{\beta}_1 = \frac{\msum x_i y_i}{\msum x_i^2}$
\end{minipage}
\begin{minipage}{0.3\textwidth}
Если:\\[2mm]
$a) \msum e_i \ne 0$\\[2mm]		
$b) \msum Y_i \ne \msum \hat{Y}_i$\\[2mm]		
$c) \bar{Y} \ne \bar{\hat{Y}},$\\[2mm]
то $R^2$ неприменим
\end{minipage}

7) Теорема Гаусса-Маркова для парной регрессии

Если:\\[-9mm]
\begin{enumerate}
	\item Модель правильно специфицирована
	\item Все $X_i$ детерминированы и не равны между собой
	\item Ошибки не носят систематического характера, т.е. $\e(\epsilon_i) =  \forall i$
	\item $\var(\epsilon_i) = \sigma_{\epsilon}^2 \forall i$
	\item Ошибки некоррелированы
\end{enumerate}

Тогд оценки МНК $\hat{\beta}_0, \, \hat{\beta}_1$ оптимальны (имеют наименьшую дисперсию) в классе линейных несмещенных оценок -- являются BLUE.

8) МНК-оценки - случайные величины
$$\hat{\beta}_1 = \frac{sCov(X,Y)}{\svar(X)} = \frac{sCov(X,\beta_0 + \beta_1 X + \epsilon)}{\svar(X)} = \beta_1 \frac{\svar(X)}{\svar(X)} + \frac{sCov(X,\epsilon)}{\svar(X)} = \beta_1 + \frac{sCov(X,\epsilon)}{\svar(X)}$$
$$\e(\hat{\beta}_1) = \e\left(\beta_1 + \frac{sCov(X,\epsilon)}{\svar(X)}\right) = \beta_1 + \frac{1}{\svar(X)} \cdot \e\left(\frac{1}{n-1} \msum(X_i - \bar{X})(\epsilon_i - \bar{\epsilon})\right) =$$
$$= \beta_1 + \frac{1}{\svar(X)} \cdot \frac{1}{n-1} \cdot \msum (X_i - \bar{X})(\e(\epsilon_i) - \e(\bar{\epsilon})) \stackrel{\e(\epsilon) = \e(\bar{\epsilon})}{=} \beta_1$$
\textbf{Утверждение 1}: $\displaystyle\var(\hat{\beta}_1) = \frac{\sigma_{\epsilon}^2}{\msum x_i^2}; \var(\hat{\beta}_0) = \frac{\sigma_{\epsilon}^2 \msum x_i^2}{n \msum x_i^2}$

$\displaystyle\hat{\beta}_1 = \frac{\isum x_i y_i}{\jsum x_j^2} = \isum \frac{x_i}{\jsum x_j} y_i = \isum w_i y_i$

\begin{minipage}{0.51\textwidth}
$a) \displaystyle\isum w_i = 0$\\

$\displaystyle\isum w_i = \frac{\isum x_i}{\jsum x_j^2} = \frac{\isum(X_i - \bar{X})}{\jsum x_j^2} = \frac{n\bar{X} - n\bar{X}}{\jsum x_j^2} = 0$
\end{minipage}
\begin{minipage}{0.18\textwidth}
$b) \displaystyle\isum w_i x_i = 1$\\[2mm]
$\displaystyle\frac{\isum x_i \cdot x_i}{\jsum x_j^2} = 1$
\end{minipage}
\begin{minipage}{0.3\textwidth}
$c) \displaystyle\isum w_i^2 = \frac{1}{\jsum x_j^2}$\\[2mm]

$\displaystyle\isum w_i^2 = \frac{\isum x_i^2}{(\jsum x_j^2)^2} = \frac{1}{\jsum x_j^2}$\\[3mm]
\end{minipage}

$$\var(\hat{\beta}_1) = \var\left(\isum w_i y_i\right) = \isum w_i^2 \var(y_i) + \isum \jsum w_i w_j Cov(y_i, y_j) = \isum w_i^2 \sigma_{\epsilon}^2 = \fbox{\text{$\displaystyle\frac{\sigma_{\epsilon}^2}{\jsum x_i^2}$}}$$
$$\hat{\beta}_0 = \bar{Y} - \hat{\beta}_1 \bar{X} = \frac{1}{n} \msum Y_i - \msum w_i (Y_i - \bar{Y}) \cdot \bar{X} = \msum \left(\frac{1}{n} - w_i \bar{X}\right) Y_i + \bar{X} \bar{Y} \msum w_i = \msum\left(\frac{1}{n} - w_i \bar{X}\right) Y_i$$
$$\e(\beta_0) = \e\left(\msum\left(\frac{1}{n} - w_i \bar{X}\right) Y_i\right) = \msum\left(\frac{1}{n} - w_i \bar{X}\right) \e(Y_i) = 0??$$
$$\var(\hat{\beta}_0) = \msum \left(\frac{1}{n^2} - \frac{2}{n} w_i \bar{X} + w_i^2 \bar{X}^2\right) \sigma_{\epsilon}^2 = \sigma_{\epsilon}^2 \left(\frac{1}{n} - \frac{2}{n} \bar{X} \msum w_i + \bar{X}^2 \msum w_i^2\right) = \sigma_{\epsilon}^2 \left(\frac{1}{n} + \frac{\bar{X}^2}{\msum x_i^2}\right) =$$
$$= \sigma_{\epsilon}^2 \frac{\msum x_i^2 + n \bar{X}^2}{n \msum x_i^2} = \fbox{\text{$\displaystyle\sigma_{\epsilon}^2 \frac{\msum X_i^2}{n \msum x_i^2}$}}$$

$\displaystyle Cov(\hat{\beta}_0, \hat{\beta}_1) - ?$

$\displaystyle\bar{Y} = \hat{\beta}_0 + \hat{\beta}_1 \bar{X}; \; \var(\bar{Y}) = \frac{\sigma_{\epsilon}^2}{n}$
$$\var(\hat{\beta}_0 + \hat{\beta}_1 \bar{X}) = \var(\hat{\beta}_0) + 2 \bar{X} Cov(\hat{\beta}_0, \hat{\beta}_1) + \bar{X}^2 \var(\hat{\beta}_1)$$
$$\frac{\sigma_{\epsilon}^2}{n} = \sigma_{\epsilon}^2 \frac{\msum X_i^2}{n \msum x_i^2} + 2 \bar{X} Cov(\hat{\beta}_0, \hat{\beta}_1) + \bar{X}^2 \frac{\sigma_{\epsilon}^2}{\jsum x_i^2}$$
$$Cov(\hat{\beta}_0, \hat{\beta}_1) = \frac{\sigma_{\epsilon}^2 \left(\frac{1}{n} - \frac{1}{n} -  \frac{\bar{X}^2}{\msum x_i^2} - \frac{\bar{X}^2}{\msum x_i^2}\right)}{2\bar{X}} = \fbox{\text{$\displaystyle\frac{- \sigma_{\epsilon}^2 \cdot 2 \bar{X}^2}{2 \bar{X} \msum x_i^2}$}}$$

\textbf{Утверждение 2}: при выполнении условией ТГМ оценки $\hat{\beta}_0, \, \hat{\beta}_1$ являются лучшими, т.е. имеют наименьшую дисперсию в классе всех линейных несмещенных оценок

Пусть $\tilde{\beta}_1 = \msum \tilde{w}_i Y_i$ -- другая несмещенная оценка, т.е. $\e(\tilde{\beta}_1) = \beta_1$
$$\e(\tilde{\beta}_1) = \msum \tilde{w}_i \e(Y_i) = \msum \tilde{w}_i \e(\beta_0 + \beta_1 X_i + \epsilon_i) = \beta_0 \msum \tilde{w}_i + \beta_1 \msum \\tilde{w}_i X_i \equiv \beta_1 \Rightarrow$$

$\displaystyle(1) \msum \tilde{w}_i = 0$

$\displaystyle(2) \msum \tilde{w}_i x_i = 1$

Т.е. необходимо решить задачу:
$\displaystyle \var(\tilde{\beta}_1) = \sigma_{\epsilon}^2 \msum \tilde{w}_i^2 \rightarrow \min$ при ограничениях (1) и (2)
$$\msum \tilde{w}_2^2 = \msum(\tilde{w}_i - w_i + w_i)^2 = \msum(\tilde{w}_i - w_i)^2 + 2\msum(\tilde{w}_i - w_i)w_i + \msum w_i^2$$
$$\isum (\tilde{w}_i - w_i) w_i = \isum \tilde{w}_i \cdot w_i - \isum w_i^2 = \isum \tilde{w}_i \frac{(X_i - \bar{X})}{\jsum x_j^2} - \isum w_i^2 = \frac{1}{\jsum x_j^2} \left(\isum \tilde{w}_i X_i - \bar{X} \msum \tilde{w}_i\right) - \frac{1}{\jsum x_j^2} =$$
$$= \frac{1}{\jsum x_j^2} - \frac{1}{\jsum x_j^2} = 0 \Rightarrow \isum \tilde{w}_i^2 = \msum(\tilde{w}_i - w_i)^2 + \frac{1}{\jsum x_j^2}$$
$\Rightarrow \msum \tilde{w}_i^2 \text{ достигает минимума при } \tilde{w}_i = w_i, \text{т.е. для оценок МНК.}$

Для $\beta_0$ доказательство аналогично
\end{document}