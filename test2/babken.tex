\documentclass[12pt]{article}
\usepackage[svgnames]{xcolor}
\usepackage{tikz} 
\usetikzlibrary{arrows} 
\usetikzlibrary{calc,positioning,fit,backgrounds} 
\usetikzlibrary{decorations.pathreplacing,calc}
\usepackage{pgfplots}
\usepackage[russian]{babel}
\parskip=6pt
\usepackage{multicol}
\usepackage{amssymb, amsmath}
\oddsidemargin=-0,5cm
\textwidth=18cm
\topmargin=-3cm
\pagestyle{plain}
\textheight=25.5cm

\newcommand{\e}{\mathbb{E}}
\newcommand{\p}{\mathbb{P}}
\newcommand{\n}{\mathbb{N}}
\newcommand{\id}{\mathbb{I}}
\newcommand{\re}{\mathbb{R}}
\DeclareMathOperator{\plim}{plim}
\DeclareMathOperator{\var}{Var}
\DeclareMathOperator{\svar}{sVar}
\renewcommand{\epsilon}{\varepsilon}
\newcommand{\msum}{\sum\limits_1^n}
\newcommand{\isum}{\sum\limits_i}
\newcommand{\jsum}{\sum\limits_j}

\begin{document}
	
\textbf{18} Тест Бокса-Кокса (для выбора функциональной формы)

Трансформация:

\begin{equation*}
Y^{(\lambda)} = 
\begin{cases}
\frac{Y^{\lambda} - 1}{\lambda}, \lambda \neq 0\\
\ln{Y}, \lambda = 0
\end{cases}
\end{equation*} 

При $\lambda = 0$:  

$\lim\limits_{\lambda \to 0} \frac{Y^{\lambda} - 1}{\lambda} = \lim\limits_{\lambda \to 0} \frac{Y^{\lambda}\ln{Y}}{1} = \ln{Y}$

Заменим:

$Y^{\lambda} = \beta_{0} + \beta_{1}X_{1}^{(\theta)} + \beta_{k-1}X_{k-1}^{(\theta)} + \epsilon$
	
\begin{equation*}
X_{i}^{(\theta)} = 
\begin{cases}
\frac{X_{i}^{(\theta)} - 1}{\theta}, \theta \neq 0\\
\ln{X_{i}}, \theta = 0
\end{cases}
\end{equation*} 

$\lambda$ и $\theta$ оцениваются через ММП

Проверка гипотез относительно $\lambda$ и $\theta$ - через LR-тест

Упрощения:

1) $\lambda = \theta$ (одинаковые преоьразования)

2) $\lambda = 1$ (преобразуются только регрессоры)

3) $\theta = 1$ (преобразуется только таргет)
	
\textbf{19} Тест Бери и МакАлера

Выбираем между полулогарифмической и линейной функцией.

Шаг 1. Найдем $\hat{\ln{Y}}$ и $\hat{Y}$

Шаг 2. Оценим вспомогательные регрессии.

$$exp(\hat{\ln{Y})} = \beta_{0} + \beta_{1}X_{1} + ... + \beta_{k-1}X_{k-1} + V_{1}$$
$$\ln{\hat{Y}} = \beta_{0} + \beta_{1}X_{1} + ... + \beta_{k-1}X_{k-1} + V_{2} $$

Отсюда находим $\hat{V_{1}}$ и $\hat{V_{2}}$

Шаг 3. Оценим модели

$$\ln{Y} = \beta_{0} + \beta_{1}X_{1} + ... + \beta_{k-1}X_{k-1} + \theta_{1}\hat{V_{1}} + \epsilon_{1}$$
$$Y = \beta_{0} + \beta_{1}X_{1} + ... + \beta_{k-1}X_{k-1} + \theta_{2}\hat{V_{2}} + \epsilon_{2}$$

Проверяем t-тестом $\theta_{1}$ и $\theta_{2}$ на значимость.
\begin{enumerate}
	\item Если $\theta_{1}$ незначим, выбираем полулогарифмическую
	\item Если $\theta_{2}$ незначим, выбираем линейную
	\item Если оба значимы/незначимы, не можем ничего выбрать
\end{enumerate}



\end{document}